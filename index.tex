
\documentclass[preprint,12pt]{elsarticle}

\usepackage[spanish]{babel}
\usepackage{amssymb}
\usepackage{graphicx}
\usepackage{lineno}
\usepackage[utf8]{inputenc}
\usepackage{url}
\usepackage{natbib}

\begin{document}
	
	\begin{frontmatter}

		\title{\huge  COMPARATIVA DE DOS METODOLOGÍAS DE CALIDAD DE SOFTWARE) }
		\author{Pacora Silva, Jorge Carlos				(2013000725)}
		\author{Quenaya Buiza, Jesús Alejandro			(20200)}
		\author{Torres Beltran , Johanna Andrea			(2020067849)}
		\address{Tacna, Perú}
		


%%INICIO abstract
\begin{abstract}
Nowadays, the companies in charge of software development are prepared to cover a wide range of demand in this market, so their objective is to cover the needs of each client. However, problems in software development such as quality are still evident.
\\
This article describes the quality methodologies that are implemented to mitigate this problem. Since, the different methods are necessary for the accomplishment of the tests to which the software is put under to identify the quality of this product that is being given to the client.
However, it is necessary to emphasize that the technology advances, everything with respect to the components of the quality evolves simultaneously, with respect to which the classic quality was left aside that was shaped by means of a paper, to happen to the agile quality where new practices are implemented, where it is possible to conclude that in this evolution it is not only enough to have good lines of code and of course it is not paper, it goes beyond, where it is supported in scopes like the productivity, the happiness and the times of delivery of the projects.
\end{abstract}
%%FIN abstract


\end{frontmatter}

%%INICIO Resumen
\section{Resumen}
En la actualidad las empresas que se encargan del desarrollo de software están preparadas para cubrir amplia gama de demanda en este mercado, por lo que su  objetivo es cubrir las necesidades que cada cliente necesite. Sin embargo, aún se evidencian problemas en el desarrollo de software como lo es la calidad.
\\
\\
En el  presente artículo se describe las metodologías de calidad que son implementadas para mitigar este problema. Puesto que, los diferentes métodos  son necesarios para la realización de las  pruebas a las que es sometido el software para identificar la calidad de este producto que se le está entregando al cliente.
Sin embargo es necesario recalcar que a la tecnología avanza, todo respecto al los componentes de la calidad  evolucionan simultáneamente, respecto a que se dejó de un lado la calidad clásica que se plasmaba mediante un papel , para pasar a la calidad ágil donde se implementan nuevas prácticas, donde se puede concluir que en esta evolución no solo basta con tener buenas líneas de código y por supuesto no es papel, va más allá, donde se apoya en ámbitos como la productividad, la felicidad y a los tiempos de entrega de los proyectos.
%%FIN Resumen


%%INICIO Introducción
\section{Introducción}
La calidad de software en los últimos años ha mejorado significativamente y se han incorporado nuevas metodologías de calidad respecto al desarrollo de software que mitigan las fallas que presentaban los antiguos estándares de calidad.
\\
En el presente artículo se explicará dos metodologías de calidad de desarrollo de software, Las cuales son la ISO/IEC 25000, tiene como objetivo crear un marco de trabajo común para evaluar la calidad del producto software y la ISO/IEC 3300,  es un proyecto de normalización que se encarga integrar una serie de normas y propone un nuevo esquema de evaluación de procesos respecto a la calidad del producto.
\\
%%FIN Introducción

%%INICIO Obejtivos
\section{Objetivos}
\begin{itemize}

	\item{\textbf{}} xxxxxxxxxxxxxxxxxxxxxxxxxxxxxxxxxxxxxxxxx.
	\item{\textbf{}} xxxxxxxxxxxxxxxxxxxxxxxxxxxxxxxxxxxxxxxxx.

\end{itemize}

%%FIN Objetivos

%%INICIO Marco Teórico
\section{Marco Teórico}

%%----------------------------------------------------------------------------------------------------------------------------------------------------------
	\subsection{\textbf{Calidad de Software}}
Según Pressman (2010) la calidad de software es el "Proceso eficaz de software que se aplica de manera que crea un producto útil que proporciona valor medible a quienes lo producen y a quienes lo utilizan"(p.340).\cite{referenciatorres1}
\\
\\
Sommerville(2005) citado por López (2015) , escribió se puede definir la calidad del software como “La concordancia con los requerimientos funcionales y de rendimiento explícitamente establecidos, con los estándares de desarrollo explícitamente documentados y con las características implícitas que se espera de todo software desarrollado profesionalmente”(p.46).\cite{referenciatorres2}
\\
\\
Pressman (2005) afirma: "La calidad de software se consigue por medio de la aplicación de métodos de ingeniería de software, prácticas adecuadas de administración y un control de calidad exhaustivo, todo lo cual es apoyado por la infraestructura de aseguramiento de la calidad. En los capítulos que siguen se estudian con cierto detalle el control y aseguramiento de la calidad"(p.352).\cite{referenciatorres1}
\\
\\
Según Pressman (2002) y Sommerville(2006) citado por López (2015), dicen que,"Para lograr la obtencion de un software de buena calidad, es importante que se realice la aplicación de procedimientos estándarizados, para las fases de análisis, diseño, programación y prueba del software. Esto con la finalidad de lograr una mayor confiabilidad, mantenibilidad y facilidad de prueba, y asi aumentar la productividad; tanto para desarrolladores de software como para el control de su calidad"(p.46).\cite{referenciatorres2}
\\
\\
	\subsection{\textbf{Metodología de Calidad de Desarrollo de Software ISO 25000}}
	Hay que tomar en cuenta que este concepto Business Intelligence es un tema que viene desde octubre de 1958.
\\
%%****

%%-----------------------------------------------------------------------------
	\subsection{\textbf{Metodología de Calidad de Desarrollo de Software ISO 33000) }}
	xxxxxxxxxxxxxxxxxxxxxxxxxxxxxxxxxxxxxxxxxxxxxxxxxxxxxxxxxxxxxxxxxxxxxxxxxxxxxxxxxxxxxxxxxxxxxxxxxxxxxxxxxxxxxxxxxxxxxxxx \\\\
Es una colección de habilidades, tecnologías, aplicaciones y prácticas para la investigación continua y repetitiva de los datos, y así poder explicar el desempeño empresarial histórico, actual y futuro. \\
\\Simplemente, los análisis convierten los datos en información útil para la empresa.

%%----------------------------------------------------------------------------------------------------------------------------------------------------------
%%FIN Marco Teórico

%COMPRACION

\section{Comparación entre ISO 25000 y ISO 3300)}
A continuacion se muestra la comparación entre la ISO 25000 Y la ISO 3300:
	
	\begin{itemize}

	\item{\textbf{1.}} 
	\item{\textbf{2.}} 
\end{itemize}


%CONCLUSIONES
\section{Conclusiones}


	\newpage
	\bibliographystyle{apalike}
	\bibliography{BIBLIOGRAFIA}	


\end{document}

